%%%%%%%%%%%%%%%%%%%%%%%%%%%%%%%%%%%%%%%%%%%%%%%%%%%%%%%%%%%%%%%%%%%%%%%%%%%%
%% Trim Size: 9.75in x 6.5in
%% Text Area: 8in (include Runningheads) x 5in
%% ws-mpla.tex   :   01-10-2021
%% TeX file to use with ws-mpla.cls written in Latex2E.
%% The content, structure, format and layout of this style file is the
%% property of World Scientific Publishing Co. Pte. Ltd.
%% Copyright 2021 by World Scientific Publishing Co.
%% All rights are reserved.
%%%%%%%%%%%%%%%%%%%%%%%%%%%%%%%%%%%%%%%%%%%%%%%%%%%%%%%%%%%%%%%%%%%%%%%%%%%%
%%

\documentclass{ws-mpla}
\usepackage[super]{cite}
\usepackage{graphicx}
\begin{document}

\markboth{Authors' names}{Instructions for typing manuscripts (paper's title)}

%%%%%%%%%%%%%%%%%%%%% Publisher's Area please ignore %%%%%%%%%%%%%%
\catchline{}{}{}{}{}
%%%%%%%%%%%%%%%%%%%%%%%%%%%%%%%%%%%%%%%%%%%%%%%%%%%%%%%%%%%%%%%%%%%

\title{Holographic dark energy models in $f(Q,T)$ gravity
}

\author{Xuwei Zhang\footnote{
Typeset names in 8 pt Times Roman. Use the footnote to
indicate the present or permanent address of the author.}}

\address{University Department, University Name, Address\\
City, State ZIP/Zone,
Country\footnote{State completely without abbreviations, the
affiliation and mailing address, including country and e-mail address.
Typeset in 8 pt Times Italic.}\\
author@emailaddress}

\author{Second Author}

\address{Group, Laboratory, Address\\
City, State ZIP/Zone, Country
}

\maketitle

\pub{Received (Day Month Year)}{Revised (Day Month Year)}

\begin{abstract}
The abstract should summarize the context, content and conclusions of
the paper in less than 200 words. It should not contain any references
or displayed equations. Typeset the abstract in 8 pt Times Roman with
baselineskip of 10 pt, making an indentation of 1.5 pica on the left
and right margins.

\keywords{Keyword1; keyword2; keyword3.}
\end{abstract}

\ccode{PACS Nos.: include PACS Nos.}

\section{Introduction} \label{sec:intro}
Over the past decaeds, a series of discoveries in cosmology have profoundly changed our understanding of the universe. In 1998, the accelerated expansion of the universe was discovered through the study of Type Ia supernovae(\cite{perlmutter_discovery_1998,Riess_1998}). This fact had been later confirmed by many other cosmological observations, such as the measurement of temperature anisotropy and polarization in the cosmic microwave background (CMB) radiation(\cite{1992ApJ...396L...1S,2020Planck}); Baryon acoustic oscillations (Baryon Acoustic Oscillations, BAO) peak length scale(\cite{Eisenstein_2005,10.1111/j.1365-2966.2011.19592.x}); the study of the large-scale structure (LSS) of the universe(\cite{Dodelson_2002,Percival_2007}) and use Cosmic Chronometers to direct measurement of Hubble parameter(\cite{Daniel_Stern_2010,Moresco_2015}). These observations suggest the existence of a mysterious energy in our universe who has high negative pressure and increasing density.

\section{$f(Q,T)$ gravity theory and Holographic dark energy}
In Weyl-Cartan geometry the connection can be decomposed into three parts: the Christoffel symbol $\hat{\Gamma}^\alpha_{\ \mu \nu}$, the contortion tensor $K^\alpha_{\ \mu \nu}$ and the disformation tensor $L^\alpha_{\ \mu \nu}$, so that the general affine connection can be expressed as
\begin{equation}
    \Gamma^\alpha_{\ \mu \nu}=\hat{\Gamma}^\alpha_{\ \mu \nu}+K^\alpha_{\ \mu \nu}+L^\alpha_{\ \mu \nu}
\end{equation}
The first term $\hat{\Gamma}^\alpha_{\ \mu \nu}$ is the Levi-civita connection of metric $g_{\mu\nu}$, given by
\begin{equation}
\hat{\Gamma}^\alpha_{\ \mu \nu}=\frac{1}{2}g^{\alpha \beta}(\partial_\mu g_{\beta \nu}+\partial_\nu g_{\beta \epsilon}-\partial_\beta g_{\epsilon \nu})
\end{equation}
The second term $K^\alpha_{\ \mu \nu}$ is the contortion tensor 
\begin{equation}
 K^{\alpha}_{\,\,\mu\nu} =\frac{1}{2}T^{\alpha}_{\,\,\mu\nu} +T^{\,\ ,\,\alpha}_{(\mu\,\,\,\nu)}
\end{equation}
The Last term is distortion tensor $L^{\alpha}_{\,\,\mu\nu}$, given by
\begin{equation}
L^{\alpha}_{\,\,\mu\nu} = \frac{1}{2}Q^{\alpha}_{\,\,\mu\nu}-Q^{\ ,\,\,\alpha}_{(\mu\,\,\,\nu)}
\end{equation}
Enclose $Q^{\alpha}_{\,\,\mu\nu}$with nonmetricity tensor
\begin{align}
Q_{\rho \mu\nu} &\equiv \nabla_{\rho} g_{\mu\nu} = \partial_\rho g_{\mu\nu} - \Gamma^\beta{}_{\rho\ mu} g_{\beta\nu} - \Gamma^\beta{}_{\rho\nu} g_{\mu\beta}~, \\
T^{\lambda}{}_{\mu\nu} &\equiv
 \Gamma^{\lambda}{}_{\mu\nu}\!-\!\Gamma^{\lambda}{}_{\nu\mu}~ ,\\
 R^{\sigma}{}_{\rho\mu\nu} &\equiv \partial_{\mu} \Gamma^{\sigma}{}_{\nu\rho}\! - \! \partial_{\nu} \Gamma^{\sigma}{}_{\mu\rho}\! +\! \Gamma^{\alpha}{}_{\nu\rho} \Gamma^{\sigma}{}_{\mu\alpha} \!-\!\Gamma^{\alpha}{}_{\mu \rho} \Gamma^{\sigma}{}_{\nu\alpha}~ ,
\end{align}

In weyl geometry, affine connection is not compatible with the metric tensor as 
\begin{equation}
    Q_{\alpha \mu \nu}=\nabla_\alpha g_{\mu \nu}=-w_\alpha g_{\mu \nu}
\end{equation}

In gravity, we consider the action is
\begin{equation}
S=\int(\frac{1}{2}f(Q,T)+\mathcal{L}_m) \sqrt{-g}  d^4x 
\end{equation}
where $f$ is an arbitrary function of the non-metricity and $T$ is the trace of the matter-energy-momentum tensor, $\mathcal{L}_m$ is known as matter Lagrangian and $g=\det (g_{\mu \nu})$ denotes determinant of metric tensor.

Vary the action we can get
\begin{align}
\delta S&=\int \left(\frac{1}{2} \delta[f(Q,T) \sqrt{-g}]+\delta(\mathcal{L}_m \sqrt{-g})\right)d^4x \\
&= \int \frac{1}{2}\left(-\frac{1}{2}f g_{\mu\nu}\sqrt{-g}\delta g^{\mu\nu}+f_Q \sqrt{-g} \delta Q+f_T \sqrt{-g}\delta T -\frac{1}{2}T_{\mu \nu}\sqrt{-g}\delta g^{\mu\nu}\right)d^4x
\end{align}
Field equation is
\begin{equation}
-\frac{2}{\sqrt{-g}}\nabla_\alpha(f_Q \sqrt{-g}P^\alpha_{\ \ \mu \nu})-\frac{1}{2}f g_{\mu \nu}+f_T(T_{\mu \nu}+\Theta_{\mu \nu})-f_Q(P_{\mu \alpha \beta}Q^\nu_{\ \ \alpha \beta}-2Q^{\alpha \beta}_{\ \ \ \ \mu}P_{\alpha \beta \nu})=T_{\mu \nu}
\end{equation}
In FLRW metric, given by

\begin{equation}
ds^2=-dt^2+a^2(t)\delta_{ij} dx^i dx^j
\end{equation}
then we can get Friedmann equations
\begin{align}
\rho &=\frac{f}{2}-6f_Q H^2-\frac{2f_T}{1+f_T}(\dot{f}_QH+f_Q \dot{H}) \\
p &=-\frac{f}{2}+6f_Q H^2+2(\dot{f}_QH+f_Q \dot{H})
\end{align}
EoS parameter
\begin{equation}
    w=\frac{p}{\rho}=-1+\frac{4 f_Q H+f_Q \dot{H}}{(1+f_T)(f-12f_QH^2)-4 f_T(\dot{f}_QH+f_Q \dot{H})}
\end{equation}
In our universe $w<-1/3$, where $\rho$ and $p$ denote total fluid energy density and pressure of the universe, which $\rho=\rho_m+\rho_{de}$, $p=p_m+p_{de}$.

Effective EoS parameter denote geometry qualities.
\begin{align}
\rho_{\text{eff}}&=3H^2=\frac{f}{4f_Q}-\frac{1}{2f_Q}[(1+f_T)\rho+f_T p]\\
-p_{\text{eff}}&=2\dot{H}+3H^2=\frac{f}{4f_Q}-\frac{2\dot{f}_Q H}{f_Q}+\frac{1}{2f_Q}[(1+f_T)\rho +(2+f_T)p]
\end{align}
EoS parameter for equivalent dark energy
\begin{equation}
w_{\text{eff}}=\frac{p_{\text{eff}}}{\rho_{\text{eff}}} = -\frac{f - 8\dot{f}_Q H + 2[(1 + f_T)\rho + (2 + f_T)p]}{f - 2[(1 + f_T)\rho + f_T p]}
\end{equation}
Combine, we can get the evolution equation of Hubble parameter $H$
Deceleration parameter
\begin{equation}
    q=-\frac{\ddot{a}a}{\dot{a}^2}=\frac{1}{2}(1+3w)=\frac{1}{2}\left(1+3\frac{p_{\text{eff}}}{\rho_{\text{eff}}}\right)=-1+\frac{3(4\dot{f}_QH-f+p)}{f-[(1+f_T)\rho+f_Tp]}
\end{equation}

We assume that 
\begin{equation}
    f(Q,T)=\alpha Q^n+\beta T
\end{equation}
where $Q=6H^2$, $T=-\rho+3p$, so that we can derive $f_Q=\alpha n Q^{n-1}=\alpha n 6^{n-1}H^{2n-2}$, $f_T=\beta$, $\dot{f}_Q=2\alpha n(n-1)6^{n-1}H^{2n-3}\dot{H}$
\section{Cosmic solutions}
Hybrid Expansion Law (HEL)
\begin{align}
     H(z)=\frac{H_0}{\sqrt{2}}\sqrt{1+(1+z)^{2m}} \\
     \dot{H}(z)=\frac{H_0}{\sqrt{2}} \frac{2m(1+z)^{2m-1}}{\sqrt{1+(1+z)^{2m}}}
\end{align}
   
use deceleration parameter to simplify calculation
\begin{equation}
    \dot{H}=-H^2(1+q)
\end{equation}
get 
\begin{align}
    &\rho=\frac{f}{2}-2f_{Q}(-H^2(1+q)+3H^2+\frac{f_T}{1+f_T}\frac{\dot{f}_Q}{f_Q}H)\\
    &p=-\frac{f}{2}+2f_Q(-H^2(1+q)+3H^2+\frac{\dot{f}_Q}{f_Q}H)
\end{align}

Holographic dark energy with Hubble cutoff is
\begin{align}
    &\rho_{de}=3c^2H^2 \\
    &p_{de}=\frac{\alpha   \left(6H(z)^2\right)^{n-1} \left(2 n (-2 \beta +(3 \beta +2) n-1) \dot{H}(z)+3 (\beta +1) (2 n-1) H(z)^2\right)}{(\beta +1) (2 \beta +1)}
\end{align}

dark energy EoS parameter
\begin{equation}
    w_{de}=\frac{p_{de}}{\rho_{de}}=\frac{\alpha \left(6H(z)^2\right)^{n-1} \left(2 n (-2 \beta +(3 \beta +2) n-1) \dot{H}(z)+3 (\beta +1) (2 n-1) H(z)^2\right)}{3c^2H^2(\beta +1) (2 \beta +1)}
\end{equation}
where $\dot{H}(z)=\frac{d}{dt}H(z)=-\frac{d H(z)}{dz}H(z)(1+z)$

stability parameter of in this situation can be described as
\begin{equation}
    c_s^2=\frac{\text{d} p_{de}}{\text{d} \rho_{de}}=\frac{\alpha(2n(-2\beta+(3\beta+2)n-1))6^{n-1}((2n-2)H^{2n-3}\dot{H}^2+\ddot{H}H^{2n-2})+\alpha3(\beta+1)(2n-1)6^{n-1}2nH^{2n-1}\dot{H}}{6c^2H \dot{H}(\beta+1)(2\beta+1)}
\end{equation}
\section{Tsalis entropy dark energy in different IR cutoff}
Tsallis proposed a modified black hole entropy
\begin{equation}
    S_{\delta}=\gamma A^{\delta}
\end{equation}
Tsallis holographic dark energy density
\begin{equation}
    \rho_{de}=BH^{4-2\delta}
\end{equation}

\section{Age of the universe}

\section{Observational data and constraint }

\section{Conclusion}
\appendix

\section{Appendix information}



\end{document} 