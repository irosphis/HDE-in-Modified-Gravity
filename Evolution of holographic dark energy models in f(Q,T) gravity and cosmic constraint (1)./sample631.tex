%% Beginning of file 'sample631.tex'
%%
%% Modified 2022 May  
%%
%% This is a sample manuscript marked up using the
%% AASTeX v6.31 LaTeX 2e macros.
%%
%% AASTeX is now based on Alexey Vikhlinin's emulateapj.cls 
%% (Copyright 2000-2015).  See the classfile for details.

%% AASTeX requires revtex4-1.cls and other external packages such as
%% latexsym, graphicx, amssymb, longtable, and epsf.  Note that as of 
%% Oct 2020, APS now uses revtex4.2e for its journals but remember that 
%% AASTeX v6+ still uses v4.1. All of these external packages should 
%% already be present in the modern TeX distributions but not always.
%% For example, revtex4.1 seems to be missing in the linux version of
%% TexLive 2020. One should be able to get all packages from www.ctan.org.
%% In particular, revtex v4.1 can be found at 
%% https://www.ctan.org/pkg/revtex4-1.

%% The first piece of markup in an AASTeX v6.x document is the \documentclass
%% command. LaTeX will ignore any data that comes before this command. The 
%% documentclass can take an optional argument to modify the output style.
%% The command below calls the preprint style which will produce a tightly 
%% typeset, one-column, single-spaced document.  It is the default and thus
%% does not need to be explicitly stated.
%%
%% using aastex version 6.3
\documentclass[linenumbers]{aastex631}
\usepackage{amsmath}

%% The default is a single spaced, 10 point font, single spaced article.
%% There are 5 other style options available via an optional argument. They
%% can be invoked like this:
%%
%% \documentclass[arguments]{aastex631}
%% 
%% where the layout options are:
%%
%%  twocolumn   : two text columns, 10 point font, single spaced article.
%%                This is the most compact and represent the final published
%%                derived PDF copy of the accepted manuscript from the publisher
%%  manuscript  : one text column, 12 point font, double spaced article.
%%  preprint    : one text column, 12 point font, single spaced article.  
%%  preprint2   : two text columns, 12 point font, single spaced article.
%%  modern      : a stylish, single text column, 12 point font, article with
%% 		  wider left and right margins. This uses the Daniel
%% 		  Foreman-Mackey and David Hogg design.
%%  RNAAS       : Supresses an abstract. Originally for RNAAS manuscripts 
%%                but now that abstracts are required this is obsolete for
%%                AAS Journals. Authors might need it for other reasons. DO NOT
%%                use \begin{abstract} and \end{abstract} with this style.
%%
%% Note that you can submit to the AAS Journals in any of these 6 styles.
%%
%% There are other optional arguments one can invoke to allow other stylistic
%% actions. The available options are:
%%
%%   astrosymb    : Loads Astrosymb font and define \astrocommands. 
%%   tighten      : Makes baselineskip slightly smaller, only works with 
%%                  the twocolumn substyle.
%%   times        : uses times font instead of the default
%%   linenumbers  : turn on lineno package.
%%   trackchanges : required to see the revision mark up and print its output
%%   longauthor   : Do not use the more compressed footnote style (default) for 
%%                  the author/collaboration/affiliations. Instead print all
%%                  affiliation information after each name. Creates a much 
%%                  longer author list but may be desirable for short 
%%                  author papers.
%% twocolappendix : make 2 column appendix.
%%   anonymous    : Do not show the authors, affiliations and acknowledgments 
%%                  for dual anonymous review.
%%
%% these can be used in any combination, e.g.
%%
%% \documentclass[twocolumn,linenumbers,trackchanges]{aastex631}
%%
%% AASTeX v6.* now includes \hyperref support. While we have built in specific
%% defaults into the classfile you can manually override them with the
%% \hypersetup command. For example,
%%
%% \hypersetup{linkcolor=red,citecolor=green,filecolor=cyan,urlcolor=magenta}
%%
%% will change the color of the internal links to red, the links to the
%% bibliography to green, the file links to cyan, and the external links to
%% magenta. Additional information on \hyperref options can be found here:
%% https://www.tug.org/applications/hyperref/manual.html#x1-40003
%%
%% Note that in v6.3 "bookmarks" has been changed to "true" in hyperref
%% to improve the accessibility of the compiled pdf file.
%%
%% If you want to create your own macros, you can do so
%% using \newcommand. Your macros should appear before
%% the \begin{document} command.
%%
\newcommand{\vdag}{(v)^\dagger}
\newcommand\aastex{AAS\TeX}
\newcommand\latex{La\TeX}

%% Reintroduced the \received and \accepted commands from AASTeX v5.2
%\received{March 1, 2021}
%\revised{April 1, 2021}
%\accepted{\today}

%% Command to document which AAS Journal the manuscript was submitted to.
%% Adds "Submitted to " the argument.
%\submitjournal{PSJ}

%% For manuscript that include authors in collaborations, AASTeX v6.31
%% builds on the \collaboration command to allow greater freedom to 
%% keep the traditional author+affiliation information but only show
%% subsets. The \collaboration command now must appear AFTER the group
%% of authors in the collaboration and it takes TWO arguments. The last
%% is still the collaboration identifier. The text given in this
%% argument is what will be shown in the manuscript. The first argument
%% is the number of author above the \collaboration command to show with
%% the collaboration text. If there are authors that are not part of any
%% collaboration the \nocollaboration command is used. This command takes
%% one argument which is also the number of authors above to show. A
%% dashed line is shown to indicate no collaboration. This example manuscript
%% shows how these commands work to display specific set of authors 
%% on the front page.
%%
%% For manuscript without any need to use \collaboration the 
%% \AuthorCollaborationLimit command from v6.2 can still be used to 
%% show a subset of authors.
%
%\AuthorCollaborationLimit=2
%
%% will only show Schwarz & Muench on the front page of the manuscript
%% (assuming the \collaboration and \nocollaboration commands are
%% commented out).
%%
%% Note that all of the author will be shown in the published article.
%% This feature is meant to be used prior to acceptance to make the
%% front end of a long author article more manageable. Please do not use
%% this functionality for manuscripts with less than 20 authors. Conversely,
%% please do use this when the number of authors exceeds 40.
%%
%% Use \allauthors at the manuscript end to show the full author list.
%% This command should only be used with \AuthorCollaborationLimit is used.

%% The following command can be used to set the latex table counters.  It
%% is needed in this document because it uses a mix of latex tabular and
%% AASTeX deluxetables.  In general it should not be needed.
%\setcounter{table}{1}

%%%%%%%%%%%%%%%%%%%%%%%%%%%%%%%%%%%%%%%%%%%%%%%%%%%%%%%%%%%%%%%%%%%%%%%%%%%%%%%%
%%
%% The following section outlines numerous optional output that
%% can be displayed in the front matter or as running meta-data.
%%
%% If you wish, you may supply running head information, although
%% this information may be modified by the editorial offices.
%\shortauthors{Schwarz et al.}
%%
%% You can add a light gray and diagonal water-mark to the first page 
%% with this command:
%% \watermark{text}
%% where "text", e.g. DRAFT, is the text to appear.  If the text is 
%% long you can control the water-mark size with:
%% \setwatermarkfontsize{dimension}
%% where dimension is any recognized LaTeX dimension, e.g. pt, in, etc.
%%
%%%%%%%%%%%%%%%%%%%%%%%%%%%%%%%%%%%%%%%%%%%%%%%%%%%%%%%%%%%%%%%%%%%%%%%%%%%%%%%%
%\graphicspath{{./}{figures/}}
%% This is the end of the preamble.  Indicate the beginning of the
%% manuscript itself with \begin{document}.

\begin{document}

\title{Holographic dark energy models in $f(Q,T)$ gravity and cosmic constraint}

%% LaTeX will automatically break titles if they run longer than
%% one line. However, you may use \\ to force a line break if
%% you desire. In v6.31 you can include a footnote in the title.

%% A significant change from earlier AASTEX versions is in the structure for 
%% calling author and affiliations. The change was necessary to implement 
%% auto-indexing of affiliations which prior was a manual process that could 
%% easily be tedious in large author manuscripts.
%%
%% The \author command is the same as before except it now takes an optional
%% argument which is the 16 digit ORCID. The syntax is:
%% \author[xxxx-xxxx-xxxx-xxxx]{Author Name}
%%
%% This will hyperlink the author name to the author's ORCID page. Note that
%% during compilation, LaTeX will do some limited checking of the format of
%% the ID to make sure it is valid. If the "orcid-ID.png" image file is 
%% present or in the LaTeX pathway, the OrcID icon will appear next to
%% the authors name.
%%
%% Use \affiliation for affiliation information. The old \affil is now aliased
%% to \affiliation. AASTeX v6.31 will automatically index these in the header.
%% When a duplicate is found its index will be the same as its previous entry.
%%
%% Note that \altaffilmark and \altaffiltext have been removed and thus 
%% can not be used to document secondary affiliations. If they are used latex
%% will issue a specific error message and quit. Please use multiple 
%% \affiliation calls for to document more than one affiliation.
%%
%% The new \altaffiliation can be used to indicate some secondary information
%% such as fellowships. This command produces a non-numeric footnote that is
%% set away from the numeric \affiliation footnotes.  NOTE that if an
%% \altaffiliation command is used it must come BEFORE the \affiliation call,
%% right after the \author command, in order to place the footnotes in
%% the proper location.
%%
%% Use \email to set provide email addresses. Each \email will appear on its
%% own line so you can put multiple email address in one \email call. A new
%% \correspondingauthor command is available in V6.31 to identify the
%% corresponding author of the manuscript. It is the author's responsibility
%% to make sure this name is also in the author list.
%%
%% While authors can be grouped inside the same \author and \affiliation
%% commands it is better to have a single author for each. This allows for
%% one to exploit all the new benefits and should make book-keeping easier.
%%
%% If done correctly the peer review system will be able to
%% automatically put the author and affiliation information from the manuscript
%% and save the corresponding author the trouble of entering it by hand.

%\correspondingauthor{August Muench}
%\email{greg.schwarz@aas.org, gus.muench@aas.org}

\author[0000-0002-0786-7307]{Xuwei Zhang}
\affiliation{Xinjiang Astronomical Observatory  \\
1667 K Street NW, Suite 800 \\
Urumqi, Xinjiang, 830046, China}

\author[0000-0002-0786-7307]{Xiaofeng Yang}
\affiliation{Xinjiang Astronomical Observatory  \\
1667 K Street NW, Suite 800 \\
Urumqi, Xinjiang, 830046, China}



\begin{abstract}

We study a holographic dark energy model in $f(Q,T)$ gravity

\end{abstract}

%% Keywords should appear after the \end{abstract} command. 
%% The AAS Journals now uses Unified Astronomy Thesaurus concepts:
%% https://astrothesaurus.org
%% You will be asked to selected these concepts during the submission process
%% but this old "keyword" functionality is maintained in case authors want
%% to include these concepts in their preprints.
\keywords{Cosmology}

%% From the front matter, we move on to the body of the paper.
%% Sections are demarcated by \section and \subsection, respectively.
%% Observe the use of the LaTeX \label
%% command after the \subsection to give a symbolic KEY to the
%% subsection for cross-referencing in a \ref command.
%% You can use LaTeX's \ref and \label commands to keep track of
%% cross-references to sections, equations, tables, and figures.
%% That way, if you change the order of any elements, LaTeX will
%% automatically renumber them.
%%
%% We recommend that authors also use the natbib \citep
%% and \citet commands to identify citations.  The citations are
%% tied to the reference list via symbolic KEYs. The KEY corresponds
%% to the KEY in the \bibitem in the reference list below. 

\section{Introduction} \label{sec:intro}
Over the past decaeds, a series of discoveries in cosmology have profoundly changed our understanding of the universe. In 1998, the accelerated expansion of the universe was discovered through the study of Type Ia supernovae(\cite{perlmutter_discovery_1998,Riess_1998}). This fact had been later confirmed by many other cosmological observations, such as the measurement of temperature anisotropy and polarization in the cosmic microwave background (CMB) radiation(\cite{1992ApJ...396L...1S,2020Planck}); Baryon acoustic oscillations (Baryon Acoustic Oscillations, BAO) peak length scale(\cite{Eisenstein_2005,10.1111/j.1365-2966.2011.19592.x}); the study of the large-scale structure (LSS) of the universe(\cite{Dodelson_2002,Percival_2007}) and use Cosmic Chronometers to direct measurement of Hubble parameter(\cite{Daniel_Stern_2010,Moresco_2015}). These observations suggest the existence of a mysterious energy in our universe, also named dark energy (DE)  who has high negative pressure and increasing density. Dark energy behaves as anti-gravity, but its nature remains unknown.

Theoretical predictions and astronomical observations indicate that there may be a mysterious form of energy in the universe. This energy has the characteristics of negative pressure, and its density increases over time. This is considered to be the key factor driving the accelerated expansion of the universe, accounting for about three-quarters of the total energy of the universe (\cite{PhysRevD.37.3406,PhysRevD.63.103510,10.1143/PTP.106.929}). In order to achieve such accelerated expansion, this form of energy needs to produce an anti-gravitational effect throughout the observable universe. However, ordinary baryonic matter neither has this equation of state nor can it explain such a large proportion of the cosmic energy component. Therefore, scientists have proposed and studied a variety of alternative theories and models to explore the nature of this cosmic acceleration phenomenon.

The simplest and most widely accepted theory is $\Lambda \text{CDM}$ model, where $\Lambda$ means cosmological constant predicted by Einstein(\cite{Carroll_2001}).
Based on $\Lambda \text{CDM}$ model, the lastest observations suggest that our universe consists of 68.3\% dark energy, 26.8\% cold dark matter and 4.9\% ordinary matter (\cite{2020Planck}). However, this model is not free from problems and the problems it is facing are cosmic coincidence, fine-tuning and the Hubble tension—a discrepancy between the value of the Hubble constant $H_0$ inferred from the CMB by the Planck satellite and that obtained from local measurements using Type Ia supernovae—has sparked significant debate. 

Another interesting attempt is to deviate from general ralativity toward a modified form (detailed research progress can be reviewid in \cite{Clifton_2012}). These theories assume that general relativity not work in large scale requiring a modification in action rather than standard Einstein-Hilbert action. The most well-known is $F(R)$ gravity which replaces the Ricci scalar $R$ in the action by a general function $f(R)$ (\cite{1970MNRAS.150....1B}). The $f(G)$ gravity theory is also a modified theory of gravity that introduces a correction to the Gauss–Bonnet (GB) term $G$, allowing it to be arbitrary function $f(G)$ rather than remaining a constant(\cite{NOJIRI20051,NOJIRI_2007}). Another modified theory of gravity $f(T)$ extends the teleparallel equivalent of General Relativity (TEGR). It replaces the curvature scalar $R$ in action with the torsion scalar $T$, derived from the Weitzenböck connection. Also shows some interpretations for the accelerating phases of our Universe(\cite{Cai_2016,Bengochea_2009}). $f(Q)$ is generalized symmetric teleparallel gravity, with curvature and torsion both being zero, which is inspired by Weyl and Einstein's trial to unify electromagenetic and gravity. The geometric properties of gravity are described by "non-metricity". That is, the covariant derivative of the metric tensor is no longer zero (some detailed information can be found in review \cite{HEISENBERG20241}). Harko et al. have proposed a new theory known as $f (R, T )$ gravity, where $R$ stands for the Ricci scalar and $T$ denotes the trace of energy-momentum tensor which presents a non-minimum coupling between geometry and matter(\cite{PhysRevD.84.024020}). Similar theories are introduced, $f(Q,T)$ proposed by Xu et al.(\cite{Xu_2019}); $f(Q,C)$ gravity(\cite{De_2024}); $f(\mathcal{T},T)$ proposed by(\cite{Harko_2014}); f(T,B) gravity(\cite{Bahamonde_2015,Bahamonde_2017}) ;$f(R,T^2)$ proposed by Katırcı et al.(\cite{Kat_rc__2014}), etc.

Holographic dark energy is an famous alternative theory for the interpretation of dark energy, originating from the holographic principle proposed by ’t Hooft(\cite{hooft2009dimensionalreductionquantumgravity}). Cohen et al. introduced the "UV-IR" relationship, highlighting that in effective quantum field theory, a system of size $L$ has its entropy and energy constrained by the Bekenstein entropy bound and black hole mass, respectively. This implies that quantum field theory is limited to describing low-energy physics outside black holes(\cite{cohen_effective_1999}). After that, Li et al. proposed that the infrared cut-off relevant to the dark energy is the size of the event horizon and obtained the dark energy density can be described as $\rho_\text{de}=3c^2 M_p^2 R_h^2$ where $R_h$ is future horizon of our universe(\cite{LI20041}). Although choose Hubble cut-off is a natural thought, but Hsu found it might lead to wrong state equation and be strongly disfavored by observational data(\cite{Hsu_2004}).

Some scholars combine or reconstruct HDE and modified theory

In this article, we assume that our universe is in a $f(Q,T)$ gravity and HDE exists objectively, we will...

\section{$f(Q,T)$ gravity theory and Holographic dark energy}
In Weyl-Cartan geometry the connection can be decomposed into three parts: the Christoffel symbol $\hat{\Gamma}^\alpha_{\ \mu \nu}$, the contortion tensor $K^\alpha_{\ \mu \nu}$ and the disformation tensor $L^\alpha_{\ \mu \nu}$, so that the general affine connection can be expressed as
\begin{equation}
    \Gamma^\alpha_{\ \mu \nu}=\hat{\Gamma}^\alpha_{\ \mu \nu}+K^\alpha_{\ \mu \nu}+L^\alpha_{\ \mu \nu}
\end{equation}
The first term $\hat{\Gamma}^\alpha_{\ \mu \nu}$ is the Levi-civita connection of metric $g_{\mu\nu}$, given by
\begin{equation}
\hat{\Gamma}^\alpha_{\ \mu \nu}=\frac{1}{2}g^{\alpha \beta}(\partial_\mu g_{\beta \nu}+\partial_\nu g_{\beta \epsilon}-\partial_\beta g_{\epsilon \nu})
\end{equation}
The second term $K^\alpha_{\ \mu \nu}$ is the contortion tensor 
\begin{equation}
 K^{\alpha}_{\,\,\mu\nu} =\frac{1}{2}T^{\alpha}_{\,\,\mu\nu} +T^{\,\ ,\,\alpha}_{(\mu\,\,\,\nu)}
\end{equation}
The Last term is distortion tensor $L^{\alpha}_{\,\,\mu\nu}$, given by
\begin{equation}
L^{\alpha}_{\,\,\mu\nu} = \frac{1}{2}Q^{\alpha}_{\,\,\mu\nu}-Q^{\ ,\,\,\alpha}_{(\mu\,\,\,\nu)}
\end{equation}
Enclose $Q^{\alpha}_{\,\,\mu\nu}$with nonmetricity tensor
\begin{align}
Q_{\rho \mu\nu} &\equiv \nabla_{\rho} g_{\mu\nu} = \partial_\rho g_{\mu\nu} - \Gamma^\beta{}_{\rho\ mu} g_{\beta\nu} - \Gamma^\beta{}_{\rho\nu} g_{\mu\beta}~, \\
T^{\lambda}{}_{\mu\nu} &\equiv
 \Gamma^{\lambda}{}_{\mu\nu}\!-\!\Gamma^{\lambda}{}_{\nu\mu}~ ,\\
 R^{\sigma}{}_{\rho\mu\nu} &\equiv \partial_{\mu} \Gamma^{\sigma}{}_{\nu\rho}\! - \! \partial_{\nu} \Gamma^{\sigma}{}_{\mu\rho}\! +\! \Gamma^{\alpha}{}_{\nu\rho} \Gamma^{\sigma}{}_{\mu\alpha} \!-\!\Gamma^{\alpha}{}_{\mu \rho} \Gamma^{\sigma}{}_{\nu\alpha}~ ,
\end{align}

In weyl geometry, affine connection is not compatible with the metric tensor as 
\begin{equation}
    Q_{\alpha \mu \nu}=\nabla_\alpha g_{\mu \nu}=-w_\alpha g_{\mu \nu}
\end{equation}

We consider the general form of the Einstein-Hilbert action for the $f(Q,T)$ gravity in the unit $8\pi G=1$
\begin{equation}
S=\int(\frac{1}{2}f(Q,T)+\mathcal{L}_m) \sqrt{-g}  d^4x \label{action}
\end{equation}
where $f$ is an arbitrary function of the non-metricity , $\mathcal{L}_m$ is known as matter Lagrangian, $g=\det (g_{\mu \nu})$ denotes determinant of metric tensor, and $T=g^{\mu \nu}T_{\mu \nu}$ is the trace of the matter-energy-momentum tensor, where $T_{\mu \nu}$ is defined as
\begin{equation}
    T_{\mu \nu}=-\frac{2}{\sqrt{-g}}\frac{\delta(\sqrt{-g}\mathcal{L}_m)}{\delta g^{\mu \nu}}
\end{equation}
Vary the action \eqref{action} with respect to the metric tensor $g_{\mu\nu}$ we can get
\begin{align}
\delta S&=\int \left(\frac{1}{2} \delta[f(Q,T) \sqrt{-g}]+\delta(\mathcal{L}_m \sqrt{-g})\right)d^4x \\
&= \int \frac{1}{2}\left(-\frac{1}{2}f g_{\mu\nu}\sqrt{-g}\delta g^{\mu\nu}+f_Q \sqrt{-g} \delta Q+f_T \sqrt{-g}\delta T -\frac{1}{2}T_{\mu \nu}\sqrt{-g}\delta g^{\mu\nu}\right)d^4x
\end{align}
Field equation is
\begin{equation}
-\frac{2}{\sqrt{-g}}\nabla_\alpha(f_Q \sqrt{-g}P^\alpha_{\ \ \mu \nu})-\frac{1}{2}f g_{\mu \nu}+f_T(T_{\mu \nu}+\Theta_{\mu \nu})-f_Q(P_{\mu \alpha \beta}Q^\nu_{\ \ \alpha \beta}-2Q^{\alpha \beta}_{\ \ \ \ \mu}P_{\alpha \beta \nu})=T_{\mu \nu}
\end{equation}
In FLRW metric, given by
\begin{equation}
ds^2=-dt^2+a^2(t)\delta_{ij} dx^i dx^j
\end{equation}
then we can get Friedmann equations
\begin{align}
\rho &=\frac{f}{2}-6f_Q H^2-\frac{2f_T}{1+f_T}(\dot{f}_QH+f_Q \dot{H}) \\
p &=-\frac{f}{2}+6f_Q H^2+2(\dot{f}_QH+f_Q \dot{H})
\end{align}
EoS parameter
\begin{equation}
    w=\frac{p}{\rho}=-1+\frac{4 f_Q H+f_Q \dot{H}}{(1+f_T)(f-12f_QH^2)-4 f_T(\dot{f}_QH+f_Q \dot{H})}
\end{equation}
In our universe $w<-1/3$, where $\rho$ and $p$ denote total fluid energy density and pressure of the universe, which $\rho=\rho_m+\rho_{de}$, $p=p_m+p_{de}$.

Effective EoS parameter denote geometry qualities.
\begin{align}
\rho_{\text{eff}}&=3H^2=\frac{f}{4f_Q}-\frac{1}{2f_Q}[(1+f_T)\rho+f_T p] \label{F1}\\
-p_{\text{eff}}&=2\dot{H}+3H^2=\frac{f}{4f_Q}-\frac{2\dot{f}_Q H}{f_Q}+\frac{1}{2f_Q}[(1+f_T)\rho +(2+f_T)p] \label{F2}
\end{align}
EoS parameter for equivalent dark energy
\begin{equation}
w_{\text{eff}}=\frac{p_{\text{eff}}}{\rho_{\text{eff}}} = -\frac{f - 8\dot{f}_Q H + 2[(1 + f_T)\rho + (2 + f_T)p]}{f - 2[(1 + f_T)\rho + f_T p]}
\end{equation}
Combine, we can get the evolution equation of Hubble parameter $H$
Deceleration parameter
\begin{equation}
    q=-\frac{\ddot{a}a}{\dot{a}^2}=\frac{1}{2}(1+3w)=\frac{1}{2}\left(1+3\frac{p_{\text{eff}}}{\rho_{\text{eff}}}\right)=-1+\frac{3(4\dot{f}_QH-f+p)}{f-[(1+f_T)\rho+f_Tp]}
\end{equation}

We assume that 
\begin{equation}
    f(Q,T)=\alpha Q^n+\beta T
\end{equation}
where $Q=6H^2$, $T=-\rho+3p$, so that we can derive $f_Q=\alpha n Q^{n-1}=\alpha n 6^{n-1}H^{2n-2}$, $f_T=\beta$, $\dot{f}_Q=2\alpha n(n-1)6^{n-1}H^{2n-3}\dot{H}$
\section{Cosmic solutions}
Use Eq.\eqref{F1} and \eqref{F2} we can solve the energy density of fluid component in field equation
\begin{align}
    \rho_\text{de}&= \frac{m 6^{n-1} (2 n-1) \left(H(t)^2\right)^{n-1} \left((3 \alpha +2) n H'(t)+3 (\alpha +1) H(t)^2\right)}{\left(2 \alpha ^2+3 \alpha +1\right) w_\text{de}} \\
    \rho_m&= \frac{m 6^{n-1} (2 n-1) \left(H(t)^2\right)^{n-1} \left(n (\alpha  (w_\text{de}-3)-2) H'(t)-3 (\alpha +1) (w_\text{de}+1) H(t)^2\right)}{\left(2 \alpha ^2+3 \alpha +1\right) w_\text{de}}
\end{align}
and the HDE energy density in Hubble cut-off can be described as
\begin{equation}
    \rho_{de}=3c^2 H(t)^2
\end{equation}
In principle, we can get the form of $H(z)$ through the solution of differential equation. However, solving higher-order differential equations analytically is difficult. So we assume $n=1$ first to simplify calculation and get 
analytically solution as follow
\begin{equation}
    H(z)= H_0 (1+z)^{\frac{3 (a+1) \left(c^2 (2 a w+w)+1\right)}{3 a+2}}
\end{equation}
In a general barrow entropy, the HDE energy density is
\begin{equation}
    \rho_{de}=3c^2 H(t)^{2-\Delta}
\end{equation}
If we set $\Delta=1$ the 
\begin{equation}
    H(z)= H_0 (1+z)^{\frac{3 (\alpha +1)}{3\alpha +2}}-(1+2 \alpha) c^2 w_{de} \left((1+z)^{\frac{3 (\alpha +1)}{3\alpha +2}}-1\right)
\end{equation}
in other situation, if $n \neq 1$ 

dark energy EoS parameter
\begin{equation}
    w_{de}=\frac{p_{de}}{\rho_{de}}=\frac{\alpha \left(6H(z)^2\right)^{n-1} \left(2 n (-2 \beta +(3 \beta +2) n-1) \dot{H}(z)+3 (\beta +1) (2 n-1) H(z)^2\right)}{3c^2H^2(\beta +1) (2 \beta +1)}
\end{equation}
where $\dot{H}(z)=\frac{d}{dt}H(z)=-\frac{d H(z)}{dz}H(z)(1+z)$

stability parameter of in this situation can be described as
\begin{equation}
    c_s^2=\frac{\text{d} p_{de}}{\text{d} \rho_{de}}=\frac{\alpha(2n(-2\beta+(3\beta+2)n-1))6^{n-1}((2n-2)H^{2n-3}\dot{H}^2+\ddot{H}H^{2n-2})+\alpha3(\beta+1)(2n-1)6^{n-1}2nH^{2n-1}\dot{H}}{6c^2H \dot{H}(\beta+1)(2\beta+1)}
\end{equation}


\section{Tsalis entropy dark energy in different IR cutoff}
Tsallis proposed a modified black hole entropy
\begin{equation}
    S_{\delta}=\gamma A^{\delta}
\end{equation}
Tsallis holographic dark energy density
\begin{equation}
    \rho_{de}=BH^{4-2\delta}
\end{equation}

\section{Observational data and parameter constraint }

\section{cosmic evolution}
The deceleration facter is defined as
\begin{equation}
    q(z)=-(1+z)H(z)\frac{d}{dz}\left(\frac{1}{H}\right)-1=(1+z)\frac{1}{H(z)}\frac{dH(z)}{dz}-1    
\end{equation}

\section{Conclusion}
\appendix

\section{Appendix information}



\bibliography{sample631}{}
\bibliographystyle{aasjournal}


\end{document}

